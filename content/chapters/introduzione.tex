\chapter{Introduzione}
\label{sec:Introduzione}

\section{Descrizione del corso}

Imprensa nell'ottica dell'amministratore delegato.

\begin{itemize}
    \item Contesto applicativo: finanza nell'ambito della strategia
    \item Le tecniche: strumenti contabili chiave
    \item Private capital, private equity, private debt, differenze con il venture capital
    \item Casi reali
\end{itemize}

\section{Risultati attesi}

\begin{enumerate}
    \item Capacitá di comprendere opportunitá e limiti per ottenere finanziamenti
    \item Capire come pensa la banca nel concedere o meno finanziamenti
    \item Capire da bilancio se c'è qualcosa che non va
    \item Conoscenza degli intermediari a livello pratico
    \item Strumenti di finanziamento equo
    \item Capacitá di formulare un piano finanziario e di leggerlo
    \item conoscenza del mondo del private capital e delle forme diverse
    \item Applicazioni
\end{enumerate}

\section{Lezioni}
Le lezioni saranno frontali con l'aggiunta di diversi workshop

\section{Libri}

Il libro consigliato fondamentale è \textit{Berk, De marzo, Corporate Finance V edition}

\section{Esame}

Scritto di 2 ore + orale facolativo \(\pm\) 3 punti.

4-5 domande + esercizi numerici + domande aperte.

\section{Introduzione}

Corporate Finance si occupa essenzialmente di 3 cose:

\begin{enumerate}
    \item Decisioni di finanziamenti
    \item Decisioni di finanziamenti
    \item Ripartizione di utili
\end{enumerate}
Noi ci occuperemo principalmente dei primi due Casi

La financa aiuta fornendo degli strumenti in un linguaggio internazionale, utili a capira la bontá di investimenti etc. etc.

i.e. un esempio di strumenti è il \textit{NPV} ovvero il \textit{Net Present Value} o \textit{Valore Attuale Netto}, utile per capire il valore attuale di investimenti futuri.

\subsection{Strategy Finance}

Parallelamente si parlerá di \textit{Strategy Finance}:

\begin{enumerate}
    \item Determinare il Current Value
    \item Identificare le opportunitá per incrementare le operazioni finanziarie efficienti
    \item Determinare quando qualche business deve essere ceduto
    \item Identificare potenziali acquisizioni e altre iniziative per stimare l'impatto sul valore
    \item Stimare come il valore delle imprese può essere accresciuto attraverso cambiamenti della struttura del capitale
\end{enumerate}

\subsection{Leva Finanziaria}
Occorre comprendere bene il concetto di \textit{leva finanziaria} in ambito di finanziamenti.

i.e. Il caso di una societá che vuole acquisire un immobile che costa 100K, gli scenari possibili sono due:

\begin{enumerate}
    \item \textbf{100\% equity}:
    
    l'impresa usa 100k di equity per acquisire l'immobile, si presentano quindi due casi a confronto:
        \subitem \textit{Upside Scenario}: Valore dell'immobile sale a 130k, il guadagno è quindi di 30k, con un incremento del 30\% dall'inizio
        \subitem \textit{Downside Scenario}: Valore dell'immobile scende a 70k, la perdita è quindi di 30k, con un decremento del -30\% dall'inizio.
    \item \textbf{50\% equity, 50\% debt}:

    l'impresa usa 50k di equity e si indebita con una banca di altri 50k per acquisire l'immobile, si presentano quindi due casi a confronto:
        \subitem \textit{Upside Scenario}: Valore dell'immobile sale a 130k, rimborsiamo 50k alla banca, rimaniamo con 80k, il guadagno è di 30k su 50k, quindi del 60\%
        \subitem \textit{Downside Scenario}: Valore dell'immobile scende a 70k, rimborsiamo 50k alla banca, rimaniamo con 20k, la perdita è di 30k su 50k, quindi del -60\%
\end{enumerate}

È chiaro quindi come ricorrere al debito sia come finanziare un progetto con una leva, sia in positivo con un incremento dei guadagni (o riduzione dell'investimento personale), sia in negativo con un incremento delle perdite

Nel caso di alti costi fissi il debito è pericoloso perchè i costi rimangono invariati anche in caso di discesa, ed essendo in leva potrebbe non essere sostenibile.

\subsection{BCG's Growth Matrix}
\begin{figure}[H]
    \centering
    \includegraphics[width=.7\linewidth]{images/introduzione/1.jpg}
    \caption{BCG's Growth Matrix}
    \label{fig:BCG}
\end{figure}

\begin{figure}[H]
    \centering
    \includegraphics[width=.7\linewidth]{images/introduzione/2.jpg}
    \caption{Esempio di cicli positivi (a sinistra) e cicli negativi (a destra)}
    \label{fig:BCGConfronto}
\end{figure}

Per questo motivo nelle Startup Innovative non si ricorre mai o quasi mai al debito, essendo progetti ad alto rischio, la probabilitá di scontrarsi con una leva negativa sono molto alte.

In una situazione di \textit{Cash Cow} ha senso indebitarsi, questo perchè con una quasi-sicurezza Coca Cola continuerà ad andare bene in modo costante. Chi fa Private Equity punta sulle \textit{Cash Cow} per il concetto di leva finanziaria sopra citato.

\subsubsection{Golden Rule}

\begin{theorem}
    Bisogna guadagnare abbastanza da un business per pagare il costo del capitale
\end{theorem}

Sembra scontato ma non lo è affatto, Alitalia non è mai riscita a rispettare questa regola e nemmeno i piccoli commercianti riescono a rispettarla, loro però riescono a sopravvivere perchè l'imprenditore, che solitamente è quasi sempre da solo, apporta lavoro proprio sottovalutandolo.

\subsubsection{Life Cycle di un'impresa, modello startup}

\begin{figure}[h]
    \centering
    \begin{minipage}[t]{.49\linewidth}
    \begin{figure}[H]
	\centering
	\includegraphics[width=\linewidth]{images/introduzione/3.jpg}
	\caption{Life Cycle investimenti}
	\label{fig:lifecycle}
    \end{figure}
    \end{minipage}
    \hfill
    \begin{minipage}[t]{.49\linewidth}
    \begin{figure}[H]
	\centering
	\includegraphics[width=\linewidth]{images/introduzione/4.jpg}
	\caption{Sorgenti di finanziamento}
	\label{fig:sorgenti}
    \end{figure}
    \end{minipage}
\end{figure}




