\chapter{Struttura del mercato finanziario e Macro Finanza}
\label{sec:Macro Finanza}

\section{Mercato finanziario}

Struttura fondamentale del sistema economico che fa delle differenze fondamentali dal mercato standard. Ad oggi è ormai assodata l'importanza del mercato finanziario (i.e. sanzioni alla Russia per la guerra in Ucraina). Un esempio è la nazionalizzazione del canale di Suez da parte di Nasser in Egitto. La nazionalizzazione del canale è stato un primo esempio di sanzione finanziaria mai perpetrata da uno stato.

In un mercato finanziario ideale:
\begin{itemize}
    \item Non ci sono costi di transazione
    \item Non ci sono tasse
    \item Tutte le attivitá sono infinitamente divisibili, senza oneri
    \item Tutti gli operatori hanno accesso simultaneamente alle stesse informazioni
    \item La possibilitá di prestare risorse ed indebitarsi sono illimitate per chiumque e avvenono allo stesso tasso di interesse.
\end{itemize}

Un moderno sistema finanziario svolge varie funzioni:
\begin{enumerate}
    \item Facilita l'allocazione di risorse da operatori in surplus a operatori in default
    \item Determina i prezzi (tassi di interesse e quotazione) a cui avvengono gli scambi
    \item Fornisce la liquiditá e commerciabilità ai titoli scambiati. La liquitdità misura quanto valore è perduto nella transazione che trasforma il titolo in denaro
    \item Gestisce la ripartizione ed il trasferimento dei rischi
    \item Fornisce la gestione dei pagamenti
    \item Raccoglie elabora e diffonde informazioni sulla societá
\end{enumerate}

Si considera il mercato come un aggregatore di informazioni.

Ci sono 3 categorie fondamentali di operatori finanziari:
\begin{itemize}
    \item \textit{Hedgers}: si corrono dei rischi grazie al mercato finanziario
    \item \textit{Arbitraggi}: vivono sulle inefficienze di breve periodo
    \item  \textit{Speculatori}: coloro che non hanno rischi ma che li cercano per poterci guadagnare
\end{itemize}

\subsubsection{Struttura base della morfologia del mercato finanziario}

\begin{figure}[H]
    \centering
    \includegraphics[width=.7\linewidth]{images/chapter1/1.jpg}
    \caption{Struttura del mercato finanziario}
    \label{fig:strutturaMercatoFinanziario}
\end{figure}

Con le inefficienze si spiegano l'esistenza delle banche e degli intermediari, perchè gli individui, recandosi in banca, delegano loro il rischio di credito e di altri rischi

Per quanto esistano gli intermediari finanziari che cecano di diversificare i rischi

\textit{Sistema bancario ombra}: intermediari che svolgono attivitá bancaria senza una regolamentazione

\subsection{Operatori e intermediari finanziari}

\subsubsection{Internazionali}

Nel 1941 la carta atlantica disegna il mondo del dopoguerra, il motivo per la stipulazione di questo trattato è il sospetto che gli inglesi volessero ingrandire il loro impero coloniale. Viene così formalmente fondato l'ONU.

Dall'ONU nascono il Fondo Monetario Internazionale e la Banca Mondiale:
\begin{itemize}
    \item \textit{Banca mondiale}: Banca che finanzia infrastrutture in paesi in via di sviluppo
    \item \textit{Fondo Monetario Internazionale}: Fondo per aiutare gli stati in difficoltá ma sotto determinate condizioni abbastanza stringenti. Assitenza a paesi che hanno bisogno di aiuto nella tenuta di fondi pubblici come la
          \href{https://it.wikipedia.org/wiki/Troika_(politica_europea)}{Troika} \footnote{\url{https://it.wikipedia.org/wiki/Troika_(politica_europea)}}. Il Fondo Monetario Europeo non donava denaro ai bisognosi ma imponeva stringenti condizioni (approfondire il \href{https://it.wikipedia.org/wiki/Washington_consensus}{Washington consensus} \footnote{\url{https://it.wikipedia.org/wiki/Washington_consensus}}
    \item \textit{BIS}: Coordina gli sforzi degli intermediari finanziari, il compito ora quello di gestire gli scambi internazionali. Andata avanti fino all'immediato secondo dopoguerra, i norvegesi chiesero di eliminare il BIS perchè accusata di detenere e nascondere ricchezze nazziste e riciclare denaro sporco. Roosvel era favorevole alla sua eliminazione, gli inglesi no per motivi ignoti, ebbero la meglio gli inglesi e l'istituzione non fu sciolta
\end{itemize}

\subsubsection{Europei}

\begin{itemize}
    \item \textit{BCE}: Banca Centrale Europea, prende decisioni per politica monetaria nell'area €
    \item \textit{EBA}: European Banking Authority, utile per le banche che non aderiscono all'euroarea. Con la BREXIT ha perso molta importanza, regolava principalmente le due monete principali, € e £. Perdendo il suo compito si è inventata un nuovo ruolo: stress test e interpretazione delle irregolaritá
    \item \textit{ESMA}: coordina le autoritá della "CONSOB" dei vari paesi
    \item \textit{EIOPA}: si occupa della vigilanza sui fondi pensione
\end{itemize}


\subsubsection{Nazionali (Italiani)}

\begin{itemize}
    \item \textit{Banca d'Italia}: Vigila sulle piccole banche nazionali e segue le direttive della banca centrale, il suo obiettivo è la \textit{Stabilitá} del sistema bancario, cosa spesso in contrasto con la competizione tra le banche
    \item \textit{IVASS}: Si occupa della regulation delle assicurazioni, è una decisione della banca d'Italia
    \item \textit{COVIP}: Vigila sui fondi pensione nazionali
    \item \textit{ANTITRUST}: Efficienza e concorrenza sul mercato bancario
    \item \textit{CONSOB}: La Commissione Nazionale per le Società e la Borsa è l'ente rivolto alla tutela degli investitori, all'efficienza, alla trasparenza e allo sviluppo del mercato mobiliare italiano \footnote{\url{https://www.consob.it/}}
\end{itemize}

\subsection{Cluster del mercato finanziario}

\begin{itemize}
    \item Ufficiale: i contratti scambiati sono identici tra di loro, i costi di transazione sono minimizzati e molto bassi, ci sono degli intermediari che garantiscono la solvibilitá
    \item Over the counter: i contratti scambiati non sono standard ma creati ad hoc, l'ingegneria finanziaria ha avuto grandissimo sviluppo in questo tipo di mercato, le possibilitá di guadagno sono infatti molto alte. La banca lavora direttamente con un cliente e crea contatti misurati direttamente sui suoi bisogni, a differenza del mercato ufficiale, in questo mercato c'è il \textit{rischio di controparte}\footnote{Il pericolo che il soggetto con cui è stato firmato il contratto si dimostri inadempiente, disattendendo i tempi e le modalità previste dal contratto stesso}. l'Over the Counter è un mercato che può sevire anche come protezione per esempio per quei mercati stagionali, come alternativa all'investimento diretto:
          \subitem \textit{Investimento diretto}: per proteggersi l'azienda acquista un'altra azienda o una parte di essa, in modo da avere introiti anche nel campo in cui è più debole
          \subitem \textit{Derivato Ad Hoc}: anziché acquistare direttamente un'azienda si può creare un derivato che quell'azienda come sottostante
\end{itemize}

Esistono due tipi di Intermediari nel mercato ufficiale:

\begin{itemize}
    \item Broker: Riceve gli ordini e li trasmette direttamente sul mercato, guadagna una commissione sull'entitá dello scambio, si tratta quindi di un intermediario puro
    \item Dealer: ha un magazzino titoli e quando un cliente si rivolge al dealer, il dealer si rivolge prima al suo magazzino titoli mettendo in comunicazione diretta due clienti, in caso in cui non si trovi la controparte allora si rivoge al mercato, il suo guadagno è dato dalla differenza di prezzo
\end{itemize}

È importante notare come il mercato più efficiente sarebbe quello popolato da Broker, che non hanno obiettivo speculativo sulla singola transazione e non hanno potere di fissare un prezzo, i prezzi sono quindi dati dal mercato, se invece il mercato fosse popolato solamente da Dealer allora il prezzo sarebbe deciso da loro in una sorta di oligopolio e sarebbe imprevedibile.

\section{Banche}

Sono l'intermediario più importante, nessun sistema finanziario può fare a meno di questa istituzione. È sufficiente pensare alle banche dal punto di vista del bilancio:

\begin{minipage}[t]{.45\linewidth}
    i \textit{depositi} sono:
    \begin{itemize}
        \item Di piccolo importo
        \item A breve termine o a vista
        \item Liquidi
        \item Certi
    \end{itemize}
\end{minipage}
\hfill
\begin{minipage}[t]{.45\linewidth}
    gli \textit{impieghi} sono:
    \begin{itemize}
        \item Di grande importo
        \item A medio-lungo termine
        \item Illiquidi
        \item Rischiosi
    \end{itemize}
\end{minipage}

Quindi la funzione della banca è quella di trasformare strumenti finanziari a breve termine in strumenti a lungo termine, di piccolo importo in strumenti di grandi capitali, strumenti certi a rischiosi, per questi motivi e altri la banca è l'infrastruttura più importante del mondo finanziario.

\textit{Bank Run}: è quel fenomeno finanziario-sociale che si manifesta con la corsa agli sportelli di moltissimi individui, quando le fragilitá della banca passano dall'essere statisticamente indipendenti all'essere correlate.

La banca è, in generale, una struttura estremamente fragile che va protetta poichè gestisce direttamente il risparmio delle famiglie.

Il fallimento di una banca ha effetti negativi molto peggiori rispetto al fallimento di un'impresa industriale. L'asset principale non sono i soldi ma è la \textit{fiducia}. L'obiettivo è evitare ad ogni costo il default delle banche, molto spesso l'errore è dei banchieri, bisognerebbe salvare le banche senza salvare i banchieri, non condannare entrambi a prescindere.

Si è istituito, per la tutela dei risparmi, il \textit{Fondo Interbancario dei Depositi} che ha l'obiettivo di evitare il \textit{Bank Run}, quando un \textit{Bank Run} si verifica l'intero sistema economico ne risente e rischia di avere conseguenze disastrose.

La banca ha essenzialmente 3 funzioni principali:
\begin{itemize}
    \item Gestisce il sistema dei pagamenti
    \item Crea moneta bancaria: è l'unico intermediario che ha la possibilitádi compiere questa attivitá
\end{itemize}

\subsubsection{Tipologie di rapporti con le banche}
Ci sono due tipologie principali di rapporti che un cliente inteso come impresa può avere con una banca:
\begin{itemize}
    \item \textit{Transaction Banking}: è il caso delle banca italiana intesa nel suo insieme, molte imprese hanno rapporti con molte banche, tutti i rapporti sono di poca entitá. Non si hanno rapporti privilegiati con le banche, si mettono le banche in concorrenza tra di loro. Se le banche fiutano un problema riguardo ad una specifica impresa non ne fanno pubblicitá, cercano infatti di ritirarsi il più velocemente possibile da qualsiasi rapporto contrattuale con lui. Il sentimento principale è quindi quello della reciproca sfiducia, ognuno fa i suoi interessi senza fidarsi della controparte, è un rapporto basato sulle garanzie
    \item \textit{Relationship Banking}: Ogni impresa ha rapporti molto approfonditi con una sola banca di riferimento che conosce a fondo la societá, le banche fanno dei sacrifici per aiutare le imprese, c'è una connessione di lungo periodo tra l'impresa e la banca
\end{itemize}

\section{Macro finanza}

La crescita economica è funzione della struttura finanziaria? Quale struttura è più efficiente: quella \textit{Bank Oriented} oppure quello \textit{Market Oriented}?

Non c'è una risposta univoca, è opinione comune che il fattore più importante, a prescindere dalla politica bancaria, è quello della \textit{Stabilitá}.

Non ci sono ruoli alternativi tra mercati e intermediari ma è chiaro che gli intermediari sono i primi a intervenire durante uno sviluppo, cosa che ovviamente la borsa non fa.

È però importante capire che intermediari e merctato devono andare insieme ed essere considerati di pari passo.

Attualmente il sistema finanziario è molto in crescita come dal grafico:

\begin{figure}[H]
    \centering
    \includegraphics[width=.7\linewidth]{images/chapter1/2.jpg}
    \caption{Andamento del sistema finanziario globale}
    \label{fig:andamentoFinanza}
\end{figure}

In italia la borsa ha un ruolo decisamente secondario. La banca è il principale finanziatore esterno di imprese, soprattutto quelle minori. Si tratta inoltre di un rapporto, nello specifico, di tipo \textit{Transaction Banking}.